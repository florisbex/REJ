\section{Conclusions and Future Work}
\label{sect:conclusion}

In this paper, we developed and implemented a framework to trace back elements of GRL models to arguments and evidence that derived from the discussions between stakeholders. We created a mapping algorithm from a formal argumentation theory to a goal model, which allows us to compute the evaluation values of the GRL IEs based on the formal semantics of the argumentation theory. 

There are many directions of future work. There are a large number of different semantics for formal argumentation, that lead to different arguments being acceptable or not. It would be very interesting to explore the effect of these semantics on goal models. Jureta \emph{et al.} develop a methodology for clarification to address issues such as ambiguity, overgenerality, synonymy, and vagueness in arguments. Atkinson \emph{et al.}~\cite{atkinson2007} define a formal set of critical questions that point to typical ways in which a practical argument can be criticized. We believe that critical questions are the right way to implement Jureta's methodology, and our framework would benefit from it. In addition, currently, we have not considered the \emph{Update} step of our framework (Figure~\ref{fig:overview}). That is, the translation from goal models to argument diagrams is still missing. The \emph{Update} step helps analysts change parts of the goal model and analyze its impact  on the underlying argument diagram. Finally, the implementation is currently a browser-based mapping from an existing argument diagramming tool to an existing goal modeling tool. By adding an argumentation component to jUCMNav, the development of goal models can be improved significantly. 