\section{Transcripts excerpts}
\label{sect:transcripts:excerpts}

\begin{table}[!htbp]
\centering
\begin{tabular}{|p{17mm}|p{63mm}|p{70mm}|}
\hline
\textit{Speaker} & \textit{Text} & \textit{Annotation}\\
\hline
(P1) & And then, we have a set of actions. Save map, open map, add and remove intersection, roads & \multirow{2}{70mm}{\textbf{[20 task (AS2)]} Student has tasks ``save map'', ``open map", ``add intersection", ``remove intersection", ``add road", ``add traffic light" \textsf{[INTRO]}}\\
\cline{1-2}
(P2) & Yeah, road. Intersection, add traffic lights &\\
\hline
(P1) & Well, all intersection should have traffic lights so it's & \multirow{4}{70mm}{\textbf{[21 critical question CQ12 for 20]} Is the task ``Add traffic light" useful/relevant?\newline
\textbf{[22 answer to 21]} Not useful, because according to the specification all intersections have traffic lights. \textsf{[ATTACK]}}\\
\cline{1-2}
(P2) & Yeah &\\
\cline{1-2}
(P1) & It's, you don't have to specifically add a traffic light because if you have &\\
\cline{1-2}
(P2)	& They need-&\\
\hline
\end{tabular}
\caption{Adding tasks, disabling useless task ``Add traffic light'' (transcript $t_1$)}
\label{table:transcripts:traffic-light}

\begin{tabular}{|p{17mm}|p{63mm}|p{70mm}|}
\hline
\textit{Speaker} & \textit{Text} & \textit{Annotation}\\
\hline
0:17:39 (P1) & And in that process there are activities like create a visual map, create a road& \textbf{[14 task (AS2)]} Student has task ``Create road'' \textsf{[INTRO]}\\
\hline
0:24:36.0 (P3)	& And, well interaction. Visualization sorry. Or interaction, I don't know. So create a visual map would have laying out roads and a pattern of their choosing. So this would be first, would be choose a pattern. & \textbf{[31 critical question CQ13 for 14]} Is Task ``Create road'' clear?\newline
\textbf{[32a answer to 31]} no, according to the specification the student should choose a pattern. \newline
\textbf{[32b answer to 31]} ``Create road'' becomes ``Choose a pattern'' \textsf{[REPLACE]}\\
\hline
0:24:55.4 (P1) &	How do you mean, choose a pattern	& \textbf{[33 critical question CQ13 for 32b]} Is ``Choose a pattern'' clear? \\
\cline{1-2}
0:24:57.5 (P3)	& Students must be able to create a visual map of an area, laying out roads in a pattern of their choosing	&\\
\cline{1-2}
0:25:07.5 (P1)	& Yeah I'm not sure if they mean that. I don't know what they mean by pattern in this case. I thought you could just pick roads, varying sizes and like, broads of roads. & 
\textbf{[34a answer to 33]} No, not sure what they mean by a pattern.\\
\cline{1-2}
0:25:26.0 (P3) & No yeah exactly, but you would have them provide, it's a pattern, it's a different type of road but essentially you would select- how would you call them, selecting a- & \multirow{3}{70mm}{\textbf{[34b answer to 33]} ``Choose a pattern'' becomes ``Choose a pattern preference'' \textsf{[REPLACE]}}\\
\cline{1-2}
0:25:36.3 (P1) & Yeah, selecting a- I don't know &\\
\cline{1-2}
0:25:38.0 (P3)	& Pattern preference maybe? As in, maybe we can explain this in the documentation &\\
\hline
0:25:43.9 (P1) & What kind of patterns though. Would you be able to select & \textbf{[35 critical question CQ13 for 34b]} Is ``Choose a pattern preference'' clear?\newline
\textbf{[36a answer to 35]} no, what kind of pattern?\\
\cline{1-2}
0:25:47.4 (P3) & Maybe, I don't know it's- & \\
\cline{1-2}
0:25:48.5 (P1)	& [inaudible] a road pattern& \multirow{2}{70mm}{\textbf{[36b answer to 35]} ``Choose a pattern preference'' becomes ``Choose a road pattern'' \textsf{[REPLACE]}}\\
\hline
\end{tabular}
\caption{Clarifying the name of a task (transcript $t_3$)}
\label{table:transcript:task-clarification}

\begin{tabular}{|p{17mm}|p{43mm}|p{90mm}|}
\hline
\textit{Speaker} & \textit{Text} & \textit{Annotation}\\
\hline
0:18:55.7 (P1) &Yeah. And then two processes, static, dynamic and they belong to the goal simulate. & \textbf{[17 goal (AS3)]} System has goal ``Simulate'' \textsf{[INTRO]}\newline
\textbf{[18 task (AS2)]} System has tasks ``Static simulation'', ``Dynamic simulation'' \textsf{[INTRO]}\newline  
\textbf{[20 decomposition (AS5)]} Goal ``Simulation" AND-decomposes into "Static simulation" and ``Dynamic simulation" \textsf{[INTRO]}\\
\hline
0:30:10.3 (P1) & 	Yeah. But this is- is this an OR or an AND? & \multirow{5}{80mm}{\textbf{[26 critical question CQ10b for 20]} Is the decomposition type of ``simulate'' correct?\newline
\textbf{[27 answer to 26]} No, it should be an OR decomposition. \textsf{[REPLACE]}}\\
\cline{1-2}
0:30:12.6 (P2) & That's and OR &\\
\cline{1-2}
0:30:14.3 (P3) & I think it's an OR &\\
\cline{1-2}
0:30:15.4 (P1) & It's for the data, it's an OR &\\
\cline{1-2}
0:30:18.1 (P3) & Yep &\\
\hline	
\end{tabular}
\caption{Incorrect decomposition type for goal \emph{Simulate} (transcript $t_3$)}
\label{table:transcript:decomposition}
\end{table}

\begin{table}[!htbp]
\centering
\begin{tabular}{|p{20mm}|p{70mm}|p{60mm}|}
\hline
Respondent & Text & Annotation\\
\hline
0:10:55.2 (P1) & Maybe developers &\textbf{[4 actor (AS0)]} Development team\\
\hline
0:11:00.8 (P2) & Development team, I don't know. Because that's- in this context it looks like she's gonna make the software & \textbf{[5 critical question CQ0 for 4]} Is actor "development team" relevant?\newline
\textbf{[6 answer to 5]} No, it looks like the professor will develop the software.\\
\hline
0:18:13.4 (P2) & I think we can still do developers here. To the system & \multirow{4}{60mm}{\textbf{[16 counter argument for 6]} According to the specification the professor doesn't actually develop the software.}\\
\cline{1-2}
0:18:18.2 (P1)  & Yeah?&\\
\cline{1-2}
0:18:19.8 (P2) & Yeah, it isn't mentioned but, the professor does-&\\
\cline{1-2}
0:18:22.9 (P1) & Yeah, when the system gets stuck they also have to be [inaudible] ok. So development team&\\	
\hline
\end{tabular}
\caption{Discussion about the relevance of an actor (transcript $t_3$)}
\label{table:transcript:irrelevant-actor}
\end{table}

\iffalse
\begin{table}[!htbp]
\begin{tabular}{|p{20mm}|p{70mm}|p{60mm}|}
\hline
Respondent & Text & Annotation\\
\hline
0:19:08.6 (P3) & Should have a link with an outsource program for the statistical distribution [inaudible] & \textbf{[21 resource (AS1)]} Actor System has resource ``Statistics library''\\
\hline
0:35:27.4 (P3) & Maybe before traffic simulation view you can- the outsource package that makes the map	& \textbf{[38 contribution (AS8)]} Resource ``Statistics library'' contributes to task ``Display traffic simulation''\\
\hline
\end{tabular}
\caption{AS1: Resource, AS8: Resource contributes to task (transcript $t_3$)}
\label{table:transcript:as1-as8}

\begin{tabular}{|p{20mm}|p{70mm}|p{60mm}|}
\hline
Respondent & Text & Annotation\\
\hline
0:15:11.2 (P1) & And then, we have a set of actions. Save map, open map, add and remove intersection, roads & \multirow{2}{60mm}{\textbf{[20 task (AS2)]} Student has tasks ``save map'', ``open map'', ``add intersection'', ``add road'', ``add traffic light'', ``remove intersection''}\\
\cline{1-2}
0:15:34.7 (P2) & Yeah, road. Intersection, add traffic lights	&\\
\hline
0:15:42.3 (P1) & Well, all intersection should have traffic lights so it's & \textbf{[21 critical question CQ*1 for 20]} Is the task ``Add traffic light'' useful/redundant? \newline
\textbf{[22 answer to 22]} Not useful, because according to the specification all intersections have traffic lights.\\
\hline
0:15:52.3 (P1) & (..) And on the technical side it's gonna be a real pain to remove one intersection you're gonna have to remove a lot more because there are only four-ways allowed and if you remove one intersection then-& \textbf{[23 critical question CQ2 for 20]} Is the task ``Remove intersection'' possible?\newline
\textbf{[24 answer to 22]} It is going to be very difficult to implement.\\
\hline
\end{tabular}
\caption{AS2: Task, CQ*1: Redundant element, CQ2: impossible task (transcript $t_1$)}
\label{table:transcript:as2-cq_star_1-cq2}

\begin{tabular}{|p{20mm}|p{70mm}|p{60mm}|}
\hline
Respondent & Text & Annotation\\
\hline
0:23:20.4 (P1) & So, sets, yeah, car influx & \textbf{[32 task (AS2)]} Student has task ``car influx''\\
\hline
0:23:41.2 (P2) & (..) If you can only control the set amount of influx from any side of this sort of random distribution, I think that is going to be less interesting than when you can say something like, this road is frequently traveled. (...) So setting it per road, I think is something we want & \textbf{[33 critical question CQ*2 on 36]} Is the task description specific/clear enough? \newline
\textbf{[34 answer to 37]} No, it is not clear where the influx is changing. Change to ``control car influx per road''\\
\hline
\end{tabular}
\caption{AS2: Task, CQ*2: Specify/clarify element (transcript $t_1$)}
\label{table:transcript:as2-cq_star_2}

\begin{tabular}{|p{20mm}|p{50mm}|p{80mm}|}
\hline
Respondent & Text & Annotation\\
\hline
0:14:31.2 (P1) & Let's see- she uses the system in her course to explain her lectures about traffic problem thing & \textbf{[11 softgoal (AS4)]} ``Explain lectures traffic theory'' (Professor)\newline
\textbf{[12 goal (AS5)]} ``Use traffic light system'' (Professor)\newline
\textbf{[13 contribution (AS7)]} ``use traffic light system'' contributes to ``explain lectures traffic theory''\\
\hline
\end{tabular}
\caption{AS4: softgoal, AS5: goal, AS7: contribution (transcript $t_3$)}
\label{table:transcript:as4-as5-as7}

\begin{tabular}{|p{20mm}|p{90mm}|p{40mm}|}
\hline
Respondent & Text & Annotation\\
\hline
0:00:10.2\newline PERSON 1 & 	So, yeah [pause] I would start with something about the context. That we have to determine who the users of the system are gonna be, stakeholders. & \textbf{[1 topic]} What are the actors?\\
\hline
\end{tabular}
\caption{AS*: Topic introduction (transcript $t_1$)}
\label{table:transcript:as-star}
\end{table}

\begin{table}[!htbp]
\centering
\begin{tabular}{|p{20mm}|p{70mm}|p{60mm}|}
\hline
Respondent & Text & Annotation\\
\hline
0:29:59.5 (P3) & (...) a variety of sequences and timing schemes should be allowed.  (...) we would have traffic light behavior gives you, I guess two options then. & \multirow{4}{60mm}{\textbf{[42 task (AS2)]} Student has task ``set sequence scheme''\newline
\textbf{[43 task (AS2)]} Student has task ``set timing scheme'' \newline
\textbf{[44 decomposition (AS10)] }Task ``set traffic light behavior'' XOR-decomposes into ``set sequence scheme'' and ''set timing scheme''}\\
\cline{1-2}
0:30:23.6 (P1) & Sequences and timing schemes &\\
\cline{1-2}
0:30:25.0 (P3) & Sequences and timing schemes. So you can either go for, yeah, sequences-&\\
\cline{1-2}
0:30:30.9 (P1) & Or timing schemes&\\
\hline	
\end{tabular}
\caption{AS2: Task, AS10: Task decomposition (transcript $t_2$)}
\label{table:transcript:as2-as10}

\begin{tabular}{|p{20mm}|p{100mm}|p{30mm}|}
\hline
Respondent & Text & Annotation\\
\hline
0:06:29.3 (P2) & So, is that a trade-off. I think so. &\multirow{2}{30mm}{\textbf{[10 negative contribution (AS11)]}  task ``generate cars randomly'' contributes negatively to softgoal ``dynamic simulation''}\\	
\cline{1-2}
0:06:36.0 (P1) & Yeah, performance versus, I don't know, functionality. Like, what you say, cars come out at the end of the map side [are generated randomly] is performance wise and, I don't know, easier to make but it is less functional. Because you can't see traffic flows that easy because, well there's fixed amount of cars so there's not really gonna be jams [the simulation is less dynamic]. Is there around Utrecht always the same amount of cars? &\\
\hline	
\end{tabular}
\caption{AS11: Negative decomposition (transcript $t_1$)}
\label{table:transcript:as11}

\begin{tabular}{|p{20mm}|p{80mm}|p{50mm}|}
\hline
Respondent & Text & Annotation\\
\hline
0:49:05.3 (P3)&So, density, speed and, is there anything else.&\multirow{3}{50mm}{\textbf{[68 critical question for 63c]} Does ``set road characteristics'' decompose into any other tasks?\newline
\textbf{[69 answer to 68]} Yes, type of cars.}\\
\cline{1-2}
0:49:20.1 (P1) & Maybe type of cars&\\	
\cline{1-2}
0:49:22.0 (P3) & Type of cars, because you could have trucks, you could have personal cars. That would be good because-&\\
\hline	
\end{tabular}
\caption{CQ: Does a task decompose into other tasks? (transcript $t_2$)}
\label{table:transcript:cq:task_decomp}

\begin{tabular}{|p{20mm}|p{60mm}|p{70mm}|}
\hline
Respondent & Text & Annotation\\
\hline

0:10:55.2 (P1) & Maybe developers or & \textbf{[4 actor (AS0)]} Development team\\
\hline
0:11:00.8 (P2)&Development team, I don't know. Because that's- in this context it looks like she's gonna make the software&\textbf{[5 critical question CQ0 for 4]} Is actor ``development team'' relevant?\newline
\textbf{[6 answer to 5]} No, it looks like the professor will develop the software.\\
\hline
..&..&..\\
\hline
0:18:13.4 (P2) & I think we can still do developers here. To the system & \multirow{2}{70mm}{\textbf{[16 counter argument for 6]} According to the specification the professor doesn't actually develop the software.}\\
\cline{1-2}
0:18:22.9 (P1)&Yeah, when the system gets stuck they also have to be [inaudible] ok. So development team&\\	
\hline	
\end{tabular}
\caption{AS0: Task, CQ0: Relevant task? CQ: Generic counterargument (transcript $t_2$)}
\label{table:transcript:as0-cq0-cq_counterarg}
\end{table}
\fi