\section{UCI Design Workshop Prompt}
\label{sect:designprompt}

\subsection*{Design Prompt: Traffic Signal Simulator}

\subsubsection*{Problem Description}

For the next two hours, you will be tasked with designing a traffic flow simulation program. Your client for this project is Professor E, who teaches civil engineering at UCI. One of the courses she teaches has a section on traffic signal timing, and according to her, this is a particularly challenging subject for her students. In short, traffic signal timing involves determining the amount of time that each of an inter- section's traffic lights spend being green, yellow, and red, in order to allow cars in to flow through the intersection from each direction in a fluid manner. In the ideal case, the amount of time that people spend waiting is minimized by the chosen settings for a given intersection's traffic lights. This can be a very subtle matter: changing the timing at a single intersection by a couple of seconds can have far- reaching effects on the traffic in the surrounding areas. There is a great deal of theory on this subject, but Professor E. has found that her students find the topic quite abstract. She wants to provide them with some software that they can use to ``play'' with different traffic signal timing schemes, in different scenarios. She anticipates that this will allow her students to learn from practice, by seeing first-hand some of the patterns that govern the subject.

\subsubsection*{Requirements}
The following broad requirements should be followed when designing this system:
\begin{enumerate}
\item
Students must be able to create a visual map of an area, laying out roads in a pattern of their choosing. The resulting map need not be complex, but should allow for roads of varying length to be placed, and different arrangements of intersections to be created. Your approach should readily accommodate at least six intersections, if not more.
\item
Students must be able to describe the behavior of the traffic lights at each of the intersections. It is up to you to determine what the exact interaction will be, but a variety of sequences and timing schemes should be allowed. Your approach should also be able to accommodate left-hand turns protected by left-hand green arrow lights. In addition:
\begin{enumerate}
\item
Combinations of individual signals that would result in crashes should not be allowed.
\item
Every intersection on the map must have traffic lights (there are not any stop signs, over- passes, or other variations). All intersections will be 4-way: there are no ``T'' intersections, nor one-way roads.
\item
Students must be able to design each intersection with or without the option to have sensors that detect whether any cars are present in a given lane. The intersection's lights' behavior should be able to change based on the input from these sensors, though the exact behavior of this feature is up to you.
\end{enumerate}
\item
Based on the map created, and the intersection timing schemes, the students must be able to simulate traffic flows on the map. The traffic levels should be conveyed visually to the user in a real-time manner, as they emerge in the simulation. The current state of the intersections' traffic lights should also be depicted visually, and updated when they change. It is up to you how to present this information to the students using your program. For example, you may choose to depict individual cars, or to use a more abstract representation.
\item
Students should be able to change the traffic density that enters the map on a given road. For ex- ample, it should be possible to create a busy road, or a seldom used one, and any variation in between. How exactly this is declared by the user and depicted by the system is up to you. Broadly, the tool should be easy to use, and should encourage students to explore multiple alternative approaches. Stu- dents should be able to observe any problems with their map's timing scheme, alter it, and see the results of their changes on the traffic patterns. This program is not meant to be an exact, scientific simulation, but aims to simply illustrate the basic effect that traffic signal timing has on traffic. If you wish, you may assume that you will be able to reuse an existing software package that provides relevant mathematical functionality such as statistical distributions, random number generators, and queuing theory.
\end{enumerate}

You may add additional features and details to the simulation, if you think that they would support these goals.

Your design will primarily be evaluated based on its elegance and clarity – both in its overall solution and envisioned implementation structure.

\subsubsection*{Desired Outcomes}

Your work on this design should focus on two main issues:

\begin{enumerate}
\item
You must design the interaction that the students will have with the system. You should design the basic appearance of the program, as well as the means by which the user creates a map, sets traffic timing schemes, and views traffic simulations.
\item
You must design the basic structure of the code that will be used to implement this system. You should focus on the important design decisions that form the foundation of the implementation, and work those out to the depth you believe is needed.
\end{enumerate}

The result of this session should be: \emph{the ability to present your design to a team of software developers who will be tasked with actually implementing it.} The level of competency you can expect is that of students who just completed a basic computer science or software engineering undergraduate degree. You do not need to create a complete, final diagram to be handed off to an implementation team. But you should have an understanding that is sufficient to explain how to implement the system to competent developers, without requiring them to make many high-level design decisions on their own.

To simulate this hand-off, you will be asked to briefly explain the above two aspects of your design after the design session is over.

\subsubsection*{Timeline}
\begin{itemize}
\item 1 hour and 50 minutes: Design session
\item 10 minutes: Break / collect thoughts
\item 10 minutes: Explanation of your design 
\item 10 minutes: Exit questionnaire
\end{itemize}