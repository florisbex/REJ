\section{RationalGRL: Logical Framework}
\label{sect:formalframework}

In Section~\ref{sect:gmas} we developed a list of critical questions and argument schemes by analyzing transcripts of discussions about the development of a traffic simulator. The resulting list is shown in Table~\ref{table:argument-schemes}. We also discussed various examples of applications of the different critical questions and all four different effects (right column of Table~\ref{table:argument-schemes}): \textsf{INTRO}, \textsf{DISABLE}, \textsf{REPLACE}, and \textsf{ATTACK}.

The examples and corresponding visualizations of the previous section provide some insight on how our RationalGRL framework can be used. However, since we would like to implement our framework in a tool, we require a more precise formalization.  In this section, we present a formalization of our framework using formal logic. This is the main approach using in formal argumentation, so it allows us to use many of the techniques developed in that area directly (see Section~\ref{sect:goalmodeling:relatedwork} for more details).

In the first two subsections we formalize a static representation of our framework: We develop a formal language to specify a GRL model (Section \ref{sect:goalmodeling:logicallanguage}) and we formalize the notion of an argument (Section \ref{sect:goalmodeling:argumentationsemantics}). In the third subsection we formalize the dynamics of our framework: we develop algorithms for instantiating argument schemes and for answering critical questions. The formal framework we develop in this section serves as the underlying model for the prototype implementation we present in the next section.

\subsection{Logic Programming}
\label{sect:logicprogramming}

Logic programming is a programming technique largely based on formal logic. Any program written in a logic programming language is a set of sentences in logical form, expressing facts and rules about some problem domain. Well-known logic programming language families include Prolog, Answer set programming (ASP) and Datalog. In all of these languages, rules are written in the form of clauses:
\begin{align*}
\texttt{H :- B1, ..., Bn.}.
\end{align*}
These clauses are understood declaratively as logical implications:
\begin{align*}
\texttt{H if B1 and ... and Bn.}.
\end{align*}
\texttt{H} is called the head of the rule and \texttt{B1, ..., Bn} is called the body. Facts are rules that have no body, and are written in the simplified form:
\begin{align}
\texttt{H.}.
\end{align}
In this article we only consider the simplest case of logic programs, where \texttt{H}, \texttt{B1}, ..., \texttt{Bn} are all atomic formulae, meaning that they are formulae without logical connectives ($\vee, \wedge, \rightarrow, \leftrightarrow$) or negation ($\neg$). Such clauses are called definite clauses or Horn clauses. 

As we saw in the example above, conjunction (``and'') is denoted with a comma. Disjunction is denoted with a semi-colon. In logic programs, \emph{variables} are usually denoted as words starting with a capital letter, while atoms start with a lower-case letter.

A simple example of a RationalGRL model as a logic program is as follows:

\begin{verbatim}
1. actor(0).
2. name(0, student).
3. softgoal(1).
4. name(1, learn).
5. has(0, 1).
6. disabled(0).
7. disabled(Y) :- actor(X), disabled(X), 
               has(X,Y).
\end{verbatim}

The first six lines of this logic program are all facts, and the last line is a rule, which means it can be used to infer new facts. The first two lines assert that identifier 0 is an actor, and that the name of this actor is ``student''. The third and fourth line assert that identifier 1 is a softgoal with name ``learn''. The fifth line assert a dependency between these two identifiers, namely that actor ``student'' has softgoal ``learn''. The sixth line states that the actor is disabled (perhaps because of some underlying argument). Finally the last line states that if \texttt{X} is an actor that is disabled, and this actor has element \texttt{Y}, then this element will also be disabled. 

If we run this program and we ask it whether element 1 is disabled, it will answer with \texttt{Yes}, since element 0 is disabled and element 0 has element 1.
>>>>>>> Stashed changes

\subsection{Logical Language for RationalGRL}
\label{sect:goalmodeling:logicallanguage}

We assume familiarity with basic notions from propositional logic and set theory. An \emph{atom} is a formula with no logical connectives ($\vee, \wedge, \rightarrow, \leftrightarrow$) or negation ($\neg$). We first define the logical language for RationalGRL. 

\begin{definition}[RationalGRL Language]
The \emph{RationalGRL language} $\mathcal{L}$ contains the following atoms:
\begin{itemize}
\item $actor(i)$: Identifier $i$ is an actor.
\item $softgoal(i)$: Identifier $i$ is a softgoal.
\item $goal(i)$: Identifier $i$ is a goal.
\item $task(i)$: Identifier $i$ is a task.
\item $resource(i)$: Identifier $i$ is a resource.
\item $decomp(k,i,J,dtype)$: Identifier $k$ is a $dtype$-decomposition (and/or/xor) from the element corresponding to identifier $i$ to the set of elements corresponding to identifiers $J$.
\item $dep(k,i,j)$: Identifier $k$ is a dependency link from the element corresponding to identifier $i$ to the element corresponding to identifier $j$.
\item $contr(k,i,j,ctype)$: Identifier $k$ is a $ctype$-contribution (+/-) from the element corresponding to identifier $i$ to the element corresponding to identifier $j$.
\item $has(i,j)$: Identifier $i$ (which is an actor) has the element corresponding to identifier $j$.
\item $disabled(i)$: The element or relationships corresponding to identifier $i$ is disabled.
\item $name(i,p)$: The name of the element corresponding to identifier $i$ is $p$.
\end{itemize}

We use integers $i,j,k\in\mathbb{N}$ as identifiers for actors and GRL elements. For all of the above definitions we assume that all identifiers used in each atom are different, i.e.,
\begin{itemize}
\item $k\not=i$,$k\not=i$,$k\not=j$
\item forall $j\in J: k\not=j, i\not=j$
\end{itemize}
\end{definition}

Using the language from Definition 1, we can define GRL models. A \emph{ground atom} is an atom without variables.

\begin{definition}[GRL model] \label{def:argument}
A \emph{GRL model} $\mathcal{M}$ is a set of ground atoms based on $\mathcal{L}$.
\end{definition}

An example of the specification of the GRL model in Figure~\ref{fig:examples:decomposition} is shown in Table~\ref{table:grl_atom_spec}, written in logic programming style. In this article we do not make use of any logic programming techniques but this would be interesting future work (see Section~\ref{sect:goalmodeling:openissues}). A more elaborate example of a specification is shown in Appendix~\ref{ch:grlspec}, showing a complete specification of the traffic simulator GRL model in Figure~\ref{fig:trafficsim}.

\begin{table}[h!]
\centering
\begin{tabularx}{0.4\textwidth}{|X|}
\hline
\texttt{goal(0).}\\
\texttt{task(1).}\\
\texttt{task(2).}\\
\texttt{name(0,simulate).}\\
\texttt{name(1,static\_simulation).}\\
\texttt{name(2,dynamic\_simulation).}\\
\texttt{decomp(3,0,[1,2],or).}\\
\hline
\end{tabularx}
\caption{Specification of the GRL model in Figure~\ref{fig:examples:decomposition}}
\label{table:grl_atom_spec}
\end{table}

\subsection{Formal argumentation}
\label{sect:goalmodeling:argumentationsemantics}

In the previous subsection we introduced an atomic language to specify a GRL model. In this subsection we give a formal definition of an argument, which is simply a set of atoms from our language.

\begin{definition}[Argument] \label{def:argument}
An \emph{argument} $A\subseteq Args$ is a set of ground atoms based on $\mathcal{L}$.
\end{definition}

This simple definition allows us to form arguments about (parts of) a GRL model. For instance, $$\{goal(0),name(0,development\_team)\},$$ is an argument stating that identifier 0 is a goal element with name ``development team''. Notice that an argument and a GRL model are essentially the same: as was discussed in Section \ref{sect:metamodel}, each element of a GRL model is an argument. 

Definition \ref{def:argument} does not put any restrictions on the content of an argument. This means it is possible to form arguments that do not correspond to correct or sensible GRL models. Consider for instance the following argument: $$\{goal(0), task(0), name(0,static\_simulation)\}.$$ This argument states that identifier 0 is both a goal and a task with name ``static simulation'', which is clearly undesirable since it does not correspond to a valid GRL model. While we could have chosen to put logical constraints on the content of an argument, we choose not to do so here. Instead, we put these restrictions on the algorithms in section~\ref{sect:algorithms}. The algorithms are constructed in such a way that they only allow the construction of arguments that correspond to valid GRL models. The advantage of this approach is that it keeps the presentation of the formal part relatively simple.

In line with the main approaches in formal argumentation, we next introduce an argumentation framework as a set of arguments and an attack relation between the arguments.

\begin{definition}[Argumentation framework~\cite{Dung1995}] \label{def:argumentationframework}
An argumentation framework $AF=(Args,Att)$ consists of a set of arguments $Args$ and an attack relationship $Att:Args\times Args$, where $(A_1,A_2)\in Att$ means that argument $A_1\in Args$ attacks arguments $A_2\in Args$.
\end{definition}

In order to determine which arguments are accepted or not, we define \emph{argumentation semantics}. We used the standard approach here, which is known as \emph{Dung's semantics}. The following notions are preliminary.

\begin{definition}[Attack, conflict-freeness, defense, and admissibility~\cite{Dung1995}] Suppose an argumentation framework $AF=(Arg,Att)$, two sets of arguments $S\cup S'\subseteq Arg$, and some argument $A\in Arg$. We say that
\begin{itemize}
\item $S$ \emph{attacks} $A$ if some argument in $S$ attacks $A$,
\item $S$ \emph{attacks} $S'$ if some argument in $S$ attacks some argument in $S'$,
\item $S$ is \emph{conflict-free} if it does not attack itself,
\item $S$ \emph{defends} $A$ if for each $B$ such that $B$ attacks $A$, $S$ attacks $B$,
\item $S$ is \emph{admissible} if $S$ is conflict-free and defends each argument in it.
\end{itemize}
\end{definition}


\begin{figure}[ht!]
\centering
\begin{tikzpicture}
        \node[minimum size=1cm] (b) [argNodeIN] at (-2,0) {$A_1$};
        \node[minimum size=1cm] (c) [argNodeIN] at (0,0) {$A_2$};
        \node[minimum size=1cm] (d) [argNodeIN] at (2,0) {$A_3$};
             
         \path
    (c) edge [attackLink] (b)
    (d) edge [attackLink] (c);
    
\end{tikzpicture}
\caption{Example argumentation framework.}
\label{fig:goalmodeling:arg2}
\end{figure}

Let us explain these definitions using the example argumentation framework in Figure~\ref{fig:goalmodeling:arg2}, where $A_2$ attacks $A_1$ and $A_3$ attacks $A_2$. There are two admissible sets: $\{A_3\}$ and $\{A_1, A_3\}$, in which $A_3$ defends $A_1$ against its attacker $A_2$. Sets containing both $A_2$ and either $A_1$ or $A_3$ are not conflict free, and the sets $\{A_1\}$ and $\{A_2\}$ do not defend themselves against $A_2$ and $A_3$, respectively. 

Given the notion of admissible sets, we can then define our argumentation semantics. There are a large number of different semantics to determine which arguments are acceptable; in this article, we choose preferred semantics.

\begin{definition}[Preferred semantics~\cite{Dung1995}] \label{def:semantics}
A preferred extension of an argumentation framework $AF$ is a maximal (w.r.t. set inclusion) admissible set of $AF$.
\end{definition}

In our example from Figure~\ref{fig:goalmodeling:arg2}, there is one preferred extension, namely $\{A_1,A_3\}$.

\subsection{Algorithms for argument schemes and critical questions}
\label{sect:algorithms}

Let us briefly take stock:
\begin{itemize}
\item We developed an initial set of argument schemes and critical questions in Section~\ref{sect:gmas}, by annotating transcripts.
\item We developed a visual notation to illustrate some argument schemes and critical questions in Section~\ref{sect:examples}.
\item We proposed a logical language consisting of atomic sentence to describe a GRL model in Section~\ref{sect:goalmodeling:logicallanguage}.
\item We formalized an ``argument'' as a set of atoms from our language, and we introduced \emph{argumentation semantics} to compute sets of accepted arguments in Section~\ref{sect:goalmodeling:argumentationsemantics}.
\end{itemize}

In this subsection we develop algorithms for applying argument schemes and critical questions. The process of applying these algorithms and generating goal models from them is shown in context of the RationalGRL methodology in Figure~\ref{fig:methodologyalgorithm} \todo{Marc}{Marc}{Create an extended version of the methodology, containing where which parts of the formal framework are used.} These algorithms produce arguments (Definition~\ref{def:argument}) and attack relations, forming an argumentation framework (Definition~\ref{def:argumentationframework}). One can then use argumentation semantics (Definition~\ref{def:semantics}) to compute sets of accepted arguments. The content of these arguments is then used to compute the resulting RationalGRL model, together with enabled and disabled GRL elements and their underlying arguments.

In the following algorithms, we assume the following global variables:

\begin{definition}[Global variables] The following variables are intended to have a \emph{global} scope and have a \emph{static} value, meaning the lifetime of the variable extends across the entire run of the program.
\begin{itemize}
\item $id$: the current highest identifier of the elements. This variable is increased for each new element that is added.
\item $Args$: contains the set of all arguments.
\item $Att$: contains the set of all attack relations.
\end{itemize}
\end{definition}


\subsubsection*{Algorithms for argument schemes}
The algorithms for the argument schemes consist of forming a new argument and adding it to the set of arguments. No attack relations are introduced.

\begin{algorithm}[h]
  \caption{Applying AS0: Actor $a$ is relevant}\label{alg:as0}
  \begin{algorithmic}[1]
    \Procedure{$AS_0$}{$a$}
    \State $id\gets id+1$\label{alg:as0:1}
    \State $A \gets \{actor(id), name(id,a)\}$\label{alg:as0:2}
    \State $Args \gets Args \cup A$\label{alg:as0:3}
    \EndProcedure
  \end{algorithmic}
\end{algorithm}

\emph{Algortihm~\ref{alg:as0} for argument scheme AS0}: The algorithm takes one argument, namely the name of the actor $a$. On line~\ref{alg:as0:1} of the algorithm, the global variable $id$ is increased by one. This ensures that each new argument has a unique identifier. On line~\ref{alg:as0:2}, the argument for the actor is formed, consisting of two statements, stating respectively that $id$ is an actor, and that the name of this actor is $a$. Finally, on line~\ref{alg:as0:3} this argument is added to the global set of arguments $Args$. In Figure~\ref{fig:examples:relevant-actor2}, the application of the argument scheme $AS_0(\text{Development team})$, results in one argument: $$Args = \{\{actor(0), name(0, \text{Development team})\}\}.$$

\begin{algorithm}[h]
  \caption{Applying AS1: Actor $a_{id}$ has resource $n$}\label{alg:as1}
  \begin{algorithmic}[1]
    \Procedure{$AS_1$}{$a_{id}, n$}
    \State $id\gets id+1$
    \State $A\gets \{resource(id), name(id,n), has(a_{id},id)\}$
    \State $Args \gets Args\cup A$
    \EndProcedure
  \end{algorithmic}
\end{algorithm}

\emph{Algorithm~\ref{alg:as1} for argument scheme AS1:} This argument scheme takes two arguments, the identifier $a_{id}$ of the actor and the resource name $n$. The algorithm is similar to the previous one, with the difference that the newly added argument contains the statement $has(a_{id},id)$ as well, meaning that the actor with id $a_{id}$ has element $id$ (which is a resource). As an example, let us formalize the argument corresponding to resource \emph{External library} of actor \emph{Traffic tycoon} in Figure~\ref{fig:transcripts:grl}. First, we assume some id is associated with the actor: $$\{actor(0),name(0,traffic\_tycoon)\}.$$ Then we can formalize the argument for the resource as follows: $$\{resource(1),name(1,external\_library),has(0,1)\}.$$

Argument scheme AS1 to AS4 are all very similar, in the sense that they all assert that some element belongs to an actor. Therefore, we only provide the algorithm for AS1 and we assume the reader can easily construct the remaining algorithms AS2-AS4.

\begin{algorithm}[h]
  \caption{Applying AS5: Goal $g_{id}$ decomposes into tasks $T_1,\ldots,T_n$}\label{alg:as5}
  \begin{algorithmic}[1]
    \Procedure{$AS_5$}{$g_{id}, \{T_1,\ldots,T_n\}, type$}
    \State $T_{id} = \emptyset$\label{alg:as5:1}
    \For{$T_i$ in $\{T_1,\ldots,T_n\}$}
      \If{$\exists_{A\in Args}\label{alg:as5:2} \{task(t_{id}),name(t_{id},T_i)\}\subseteq A$}
        \State $T_{id} \gets T_{id} \cup \{t_{id}\}$
      \Else\label{alg:as5:3}
        \State $id\gets id+1$
        \State $A \gets \{task(id),name(id,T_i)\}$
        \State $Args \gets Args\cup A$
        \State $T_{id} \gets T_{id} \cup \{id\}$
      \EndIf
    \EndFor
    \State $id\gets id+1$\label{alg:as5:4}
    \State $A\gets \{decomp(id, g_{id}, T_{id}, type)\}$
    \State $Args \gets Args\cup A$
    \EndProcedure
  \end{algorithmic}
\end{algorithm}

\emph{Algorithm~\ref{alg:as5} for argument scheme AS5:} The procedure in Algorithm~\ref{alg:as5} takes three arguments: $g_{id}$ is the identifier of goal $G$, $T=(T_1,\ldots,T_n)$ is a list of decomposing task names, and $type\in\{and,or,xor\}$ is the decomposition type. The difficulty of this algorithm is that each of the tasks are stated in natural language, and it is not directly clear whether these tasks are already in the GRL model or not. Therefore, we have to check for each tasks whether it already exists, and if not, we have to create a new task. On line~\ref{alg:as5:1}, the set $T_{id}$ is initialized, which will contain the ids of the tasks $T_1,\ldots,T_n$ to decompose into. In the for-loop, the if-statement on line~\ref{alg:as5:2} checks whether some argument already exists for the current task $T_i$, and if so, it simply adds the identifier of the task ($t_{id}$) to the set of task identifiers $T_{id}$.\footnote{Note that the existing task may be disabled, but this does not matter, since the decomposition relation will be suppressed for disabled elements.} Otherwise (line~\ref{alg:as5:3}), a new task is created and the new identifier $id$ is added to the set of task identifiers. After the for loop on line~\ref{alg:as5:4}, an argument for the decomposition link itself is constructed, and it is added to the set of arguments $Args$.

Let us explain this algorithm with the XOR-decomposition of goal \texttt{Generate cars} of Figure~\ref{fig:transcripts:grl}. Suppose the following arguments are constructed already:
\begin{itemize}
\item $\{goal(0),name(0,generate\_cars)\},$
\item $\{taks(1),name(1,keep\_same\_cars\}.$
\end{itemize}
Suppose furthermore that someone wants to put forward the argument that goal \texttt{Generate cars} XOR-decomposes into tasks \texttt{Keep same cars} and \texttt{Create news cars}. Formally: $AS_5(0,\{generate\_cars,keep\_same\_cars\},xor)$. The algorithm will first set $T_{id}=\emptyset$, and then iterate over the two task names. For the first task $generate\_cars$, there does not exist an argument $\{task(t_{id}),name(t_{id},generate\_cars)\}$ yet, so a  new argument is created. Suppose the following argument is created: $\{task(2),name(2,generate\_cars)\}.$ After this, 2 is added to $T_{id}$. For the second task an argument exists already, namely $\{task(1),name(1,keep\_same\_cars)\}$, so 1 is simply added to $T_{id}$. After the for loop, we have $T_{id}=\{1,2\}$. Next, the decomposition argument is created, which is $\{decomp(3,0,\{1,2\},xor)\}$. This argument is added to $Args$ and the algorithm terminates.

\begin{algorithm}[h]
  \caption{Applying AS6: Task $t_{id}$ contributes to softgoal $s$}\label{alg:as6}
  \begin{algorithmic}[1]
    \Procedure{$AS_6$}{$t_{id}, s$}
    \If{$\exists_{A\in Args} \{softgoal(i),name(i,s)\}\subseteq A$} \label{alg:as6:if}
        \State $s_{id} \gets i$
    \Else
      \State $id\gets id+1$
      \State $A \gets \{softgoal(id),name(id,t)\}$
      \State $Args \gets Args\cup A$
      \State $s_{id} \gets id$
    \EndIf
    \State $id\gets id+1$
    \State $A\gets \{contr(id, t_{id}, s_{id}, pos)\}$
    \State $Args \gets Args\cup A$
    \EndProcedure
  \end{algorithmic}
\end{algorithm}

\emph{Algorithm~\ref{alg:as6} for argument scheme AS6:} The procedure in Algorithm~\ref{alg:as6} takes two arguments: $t_{id}$ is the identifier of task $T$, and $s$ is the softgoal name that is contributed to. The idea behind this algorithm is very similar to the previous one, with the difference that in the current algorithm we create a single relation, while we created a set of relations in the previous algorithm. First, the if-statement on line~\ref{alg:as6:if} checks whether the softgoal exists already, and if not, an argument is added for it. This ensures that all softgoals have corresponding arguments. After the if-statement, the argument for the contribution link is created and it is added to the set of arguments $Args$. 

Let us again illustrate this with a simple example from Figure~\ref{fig:transcripts:grl}. Suppose the following argument exists already: $\{task(0),name(0,keep\_same\_cars\}$, and suppose someone would like to add an argument that the task \texttt{Keep same cars} contributes positively to softgoal \texttt{Dynamic simulation}, i.e. $AS_6(0,dynamic\_simulation)$. The algorithm first checks whether an argument already exists for the softgoal, and when it finds out it does not exist, creates the argument $\{softgoal(1),name(1,dynamic\_simulation)\}$ and adds it to $Args$. Then, the argument for the contribution is added to $Args$ as well, which is $\{contr(2,0,1,pos)\}$.

\emph{Algorithms for argument schemes AS7-AS11:} The algorithms for $AS7$ to $AS11$ all have a very similar structure as Algorithm~\ref{alg:as6} and have therefore been omitted. Again, we assume the reader can reconstruct them straightforwardly.

\subsubsection*{Algorithms for critical questions}

We now develop algorithms for our critical questions. Recall that answering a critical question can have four effects, and we discuss each of these effects separately.

\begin{algorithm}[h]
  \caption{Applying DISABLE: Element $i$ is disabled}\label{alg:disable}
  \begin{algorithmic}[1]
    \Procedure{DISABLE}{$i$}
    \State $id\gets id+1$
    \State $A\gets \{disabled(i))\}$
    \State $Args \gets Args\cup A$
    \EndProcedure
  \end{algorithmic}
\end{algorithm}

\emph{Algorithm~\ref{alg:disable} (\textsf{DISABLE}) for critical questions CQ0-CQ5a, CQ6a, CQ7a, CQ8, CQ11, and CQ12:} The disable operation consists of adding an argument stating the GRL element with identifier $i$ is disabled. Let us reconsider the example of Figure~\ref{fig:examples:relevant-actor}. This example consists of an instantiation of argument scheme AS0, which is attacked by an argument that resulted from answering critical question. The instantiation of AS0 leads to the argument $A=\{actor(0),name(0,dev\_team\}$. After applying $DISABLE(0)$ we obtain the arguments: $$Args=\{\{actor(0),name(0,dev\_team\}, \{disabled(0)\}\}.$$ Note that our implementation of this attack does not lead to an actual attack in the argumentation framework.

\begin{algorithm}[h]
  \caption{Answering CQ5b: ``Does goal $G$ decompose into any other tasks?'' With: ``Yes, namey into tasks $t_1,\ldots,t_k$''}\label{alg:cq5b}
  \begin{algorithmic}[1]
    \Procedure{CQ5b}{$g_{id},\{i_1,\ldots,i_n\}, type, \{t_1,\ldots,t_k\}$}
    \State $T_{id} = \{i_1,\ldots,i_n\}$
    \For{$t_i$ in $\{t_1,\ldots,t_k\}$}\label{alg:cq5b:for1}
      \If{$\exists_{A\in Args} \{task(t_{id}),name(t_{id},t_i)\}\subseteq A$}
        \State $T_{id} \gets T_{id} \cup \{t_{id}\}$
      \Else
        \State $id\gets id+1$
        \State $A \gets \{task(id),name(id,t_i)\}$
        \State $Args \gets Args\cup A$
        \State $T_{id} \gets T_{id} \cup \{id\}$
      \EndIf
    \EndFor
    \State $id\gets id+1$
    \State $A_{new} = \{decomp(id, g_{id}, T_{id}, type)\}$\label{alg:cq5b:newarg}
    \For{$A$ in $\{decomp(\_,g_{id},\_,\_)\}\subseteq A\mid A\in Args\}$}\label{alg:cq5b:for2}
      \State $Att\gets Att \cup \{(A_{new},A)\}$
    \EndFor
    \State $Args \gets Args \cup \{A_{new}\}$
    \EndProcedure
  \end{algorithmic}
\end{algorithm}

\emph{Algorithm~\ref{alg:cq5b} (\textsf{REPLACE}) for critical questions CQ5b:} This algorithm is executed when critical question CQ5b is answered, which is a critical question for argument scheme AS5. Therefore, it assumes an argument for a goal decomposition already exists of the following form (see Algorithm~\ref{alg:as5}): $$\{decomp(d, g_{id}, \{i_1,\ldots,i_n\},type).$$  The goal of the algorithm is to generate a new argument of the form $decomp(d, g_{id}, \{i_1,\ldots,i_5\}\cup \{j_1,\ldots,j_k\}, type)$, where $\{j_1,\ldots,j_k\}$ are the identifiers of the additional decomposing tasks $\{t_1,\ldots,t_k\}$.

The algorithm takes as input the goal identifier $g_{id}$, the set of existing decomposing task identifiers $i_1,\ldots,i_n$, the decomposition type, and the names of the new tasks $t_1,\ldots,t_k$ that should be added to the decomposition. The first part of the algorithm is familiar from Algorithm~\ref{alg:as5}: For each task name we check whether it already exists as an argument (line 4), and if it doesn't (line 6) we add a new argument for it. After the for-loop (line 13), a new argument is created for the new decomposition relation (14). After this, the for-loop on line 15 ensures that the new argument attacks all previous arguments for this decomposition (note that the variable ``$\_$'' means ``do not care''). Only at the very end the new argument is added (line 18), to ensure it does not attack itself after the for loop of line 15-17. 

An example of this algorithm is shown in Figure~\ref{fig:examples:cq5b}.\footnote{Note that part of the arguments (the statements about actors) have been omitted from the figure for readability.} Before the critical question is applied, the following arguments have been put forward:
\begin{itemize}
\item $\{goal(0),name(0,show\_simulation)\}$
\item $\{task(1),name(1,generate\_traffic)\}$
\item $\{task(2),name(2,compute\_lights)\}$
\item $\{decomp(3,0,\{1,2\},and)\}$.
\end{itemize}

Next, Algorithm~\ref{alg:cq5b} is called as follows: $CQ5b(0,\{1,2\},and,\{show\_controls\})$. That is, the existing decomposition is challenged by stating that goal $show\_simulation$ not only decomposes into $generate\_traffic$ and $compute\_lights$, but it also decomposes into $show\_controls$. Since this task does not exist yet, it is created by the algorithm, which also ensures the new argument for the decomposition link attacks the previous argument for the decomposition link.

\begin{figure*}[ht!]
\centering
        \begin{tikzpicture}
        \node (a0) [argNodeIN,minimum width=1.5cm] at (-5,0) {
        	\smallargtext{AS3}{$\{goal(0),$\\$name(0,show\_sim\}$}
        };
        \node (a1) [argNodeIN] at (0,0) {
        \smallargtext{AS2}{$\{task(1),$\\$name(1,gen\_traffic\}$}
        };
        \node (a2) [argNodeIN] at (5,0) {
        \smallargtext{AS2}{$\{task(3),$\\$name(2,show\_ctrls\}$}
        };
        \node (a2a) [argNodeIN] at (-2.5,-2) {
        \smallargtext{AS2}{$\{task(2),$\\$name(2,comp\_lights\}$}
        };
        \node (a3) [argNodeOUT] at (-3,-4) {$\{decomp(3,0,\{1,2\},and\}$
        };
        \node (a4) [argNodeIN] at (3,-4) {$\{decomp(3,0,\{1,2,3\},and\}$
        };
         
         \path
    (a4) edge [attackLink] (a3)
    (a3) edge [CQLink, bend right=50] node [below,draw=none] {CQ5b} (a4)
    (a4) edge [CQLink] node [left,draw=none] {CQ5b} (a2);
\end{tikzpicture}
\caption{Example of applying critical question CQ5b (Algorithm~\ref{alg:cq5b})}
\label{fig:examples:cq5b}
\end{figure*}

\emph{Algorithms for critical questions CQ10a and CQ10b (REPLACE)}: These algorithms have a very similar structure as Algorithm~\ref{alg:cq5b} and have therefore been omitted.

\begin{algorithm}[h]
  \caption{Answering CQ13: ``Is the name of element $i$ clear?'' With: ``No, it should be $n$''}\label{alg:cq13}
  \begin{algorithmic}[1]
    \Procedure{CQ13}{$i, n$}
    \State $ArgsN \gets\{ A\in Args \mid name(i,x)\in A\}$
    \State $B\gets B'\backslash \{name(i,\_)\}$ with $B'\in ArgsN$
    \State $A \gets B \cup \{name(i,n)\}$
    \State $Args \gets Args \cup \{A\}$
    \For{$C$ in $ArgsN$}
      \State $Att\gets Att \cup \{(A,C)\}$
    \EndFor
    \EndProcedure
  \end{algorithmic}
\end{algorithm}

\emph{Algorithms for critical question CQ13 (REPLACE):} This algorithm is used to clarify/change the name of an element. It takes two parameters: the element identifier $i$ and the new name $n$. The idea behind the algorithm is that we construct a new argument for $n$, and to ensure that this argument attacks all previous arguments for giving a name to this element. Here we can use the following fact: Suppose $Args$ is a set of arguments, each containing an atom about the name for an element $i$, i.e. for all $A\in Args: name(i,_)\in Args$, then all arguments in $Args$ are the same except for the $name$ atoms, i.e. for all $A,B\in Args: A\backslash\{name(i,_)\} = B\backslash\{name(i,_)\}$. This means that if we would like to attack every argument of $Args$ with a new argument that replaces the $name$ atom, we can simply take any argument in $Args$, remove the previous $name$ atom, add the new one and attack all arguments in $Args$. This is exactly what the algorithm does.


On line 2, all arguments that have been put forward for this element and contain $name(i,x)$ are collected into the set $ArgsN$. On line 3, some arguments $B'\in ArgsN$ minus the $name$ statement is assigned to $B$ (note that it does not matter which one we pick), and on line 4 $B$ is joined with the new $name$ statement and stored in $A$, which is then added to the set of arguments $Args$. The for-loop on lines 6-8 ensures all previous arguments for names of the element are attacked by the new argument.

\begin{figure}[ht!]
\centering
        \begin{tikzpicture}[->]
        \node (a0a) [argNodeIN] at (-2,2) {
        	\longargtext{AS0}{$\{actor(0),name(0,student)\}$}
        };
        \node (a0) [argNodeOUT] at (-2,0) {
        	\longargtext{AS2}{$\{task(1),name(1,create\_road),has(0,1)\}$}
        };
        \node (a1) [argNodeOUT] at (-2,-2) {
        	\longargtext{AS2}{$\{task(1),name(1,choose\_pattern),has(0,1)\}$}
        };
        \node (a2) [argNodeOUT] at (-2,-4) {
        	\longargtext{AS2}{$\{task(1),name(1,pattern\_pref),has(0,1)\}$}
        };
        \node (a3) [argNodeIN] at (-2,-6) {
        	\longargtext{AS2}{$\{task(1),name(1,road\_pattern),has(0,1)\}$}
        };
\begin{pgfonlayer}{background}
         \path
    (a1) edge[attackLink] (a0)
    (a2) edge[attackLink] (a1)
    (a2) edge[attackLink, bend right=20] (a0)
    (a3) edge[attackLink] (a2)
    (a3) edge[attackLink, bend right=20] (a1)
    (a3) edge[attackLink, bend right=40] (a0);
\end{pgfonlayer}

	\path
	(a0) edge [CQLink, bend right=50] node [left,draw=none] {CQ13} (a1)
    (a1) edge [CQLink, bend right=50] node [left,draw=none] {CQ13} (a2)
    (a2) edge [CQLink, bend right=50] node [left,draw=none] {CQ13} (a3);
\end{tikzpicture}
\caption{Applying critical question CQ13 (Algorithm~\ref{alg:cq13}) to the example in Figure~\ref{fig:examples:clarification}.}
\label{fig:examples:clarification:formal}
\end{figure} 

An example of of Algorithm~\ref{alg:cq13} is shown in Figure~\ref{fig:examples:clarification:formal}. Let us consider the last application of CQ13 (bottom argument). Before this application, the following arguments have been put forward:
\begin{itemize}
\item $A_1$: $\{actor(0),name(0,student)\}$
\item $A_2$:$\{task(1),name(1,create\_road),has(0,1)\}$
\item $A_3$ $\{task(1),name(1,choose\_pattern),has(0,1)\}$
\item $A_4$:$\{task(1),name(1,pattern\_pref),has(0,1)\}$
\end{itemize}
The algorithm is now called as follows: $CQ13(1,road\_pattern)$, i.e., the new name of the element should be $road\_pattern$. Let us briefly run through the algorithm. After executing line 2 we obtain $ArgsN=\{A_2,A_3,A_4\}$, since only those arguments contain $name(1,\_)$. Next, on line 3, $B=\{task(1),has(0,1)\}$, i.e., $B$ is the general argument for the task without the $name$ statement. After line 4 we have $$A=\{task(1),has(0,1),name(1,road\_pattern),$$ which is added to $Args$ and attacks arguments $A_2,A_3$, and $A_4$. 

\emph{Algorithms for critical questions CQ6b, CQ6c, CQ6d, CQ7b, and CQ9 (INTRO):} The introduction algorithms for the critical questions are all very similar to the INTRO algorithms for argument schemes (Algorithm~\ref{alg:as1}). They have therefore been omitted.

\begin{algorithm}[h]
  \caption{Generic counterargument to argument $A$}\label{alg:cq6b}
  \begin{algorithmic}[1]
    \Procedure{Attack}{$A$}
    \State $A_{new} = \{\}$
    \State $Args \gets Args \cup \{A_{new}\}$
    \State $Att \gets Att \cup \{(A_{new},A)\}$
    \EndProcedure
  \end{algorithmic}
\end{algorithm}

\emph{Algorithm for $Att$ (Generic counter argument:} Applying a generic counter argument is very simple, and simply results on an attack on the original argument. We illustrate this by continuing our example from Figure~\ref{fig:examples:relevant-actor2:formal} (Algorithm~\ref{alg:as0}). The example is shown in Figure~\ref{fig:examples:relevant-actor2:formal}, where we see that a generic counter argument simply attacks the argument to disable the actor.

\begin{figure}[ht!]
\centering
        \begin{tikzpicture}
        \node (a0) [argNodeIN] at (-3.7,0) {
        	\smallargtext{AS0}{$\{actor(0),name(0,dev\_team\}$}
        } ;
        \node (a1) [argNodeOUT] at (-3.7,-2){
        	\argtext{DISABLE}{$\{disabled(0)\}$}
        };
        \node (a2) [argNodeIN] at (-3.7,-4){
        	\argtext{Att}{$\{\}$}
        };
         \path
    (a0) edge [CQLink, bend right=50] node [left,draw=none] {CQ0} (a1)
    (a1) edge [CQLink, bend right=50] node [left,draw=none] {Att} (a2)
    (a2) edge [attackLink] (a1);
\end{tikzpicture}
\caption{Formalization of the arguments in Figure~\ref{fig:examples:relevant-actor2}.}
\label{fig:examples:relevant-actor2:formal}
\end{figure}

\subsection{Constructing GRL models}

\todo{F}{F,M}{It would be good to have a formal definition of how to construct models given an argumentation framework. If we define a model M at the start of this section, this should be fairly easy.}

Constructing GRL models from the arguments is extremely simple: We simply compute the extensions of the argumentation frameworks, and collect all atomic sentences in the accepted arguments. This forms out GRL model. Let us briefly do so for the examples of the previous subsection:
\begin{itemize}
\item
Figure~\ref{fig:examples:relevant-actor:formal}: Since there are no attacks between the arguments, all atomic sentences are accepted. This results in the following specification: 
\begin{quote}
\begin{verbatim}
actor(0).
name(0,dev_team).
disabled(0).
\end{verbatim}
\end{quote}
This again corresponds to the GRL model on the right-hand side of Figure~\ref{fig:examples:relevant-actor}.
\item
Figure~\ref{fig:examples:cq5b}: There is one rejected argument and five accepted ones. The resulting specification is:
\begin{quote}
\begin{verbatim}
goal(0). name(0,show_simulation).
task(1). name(1,generate_traffic).
task(2). name(2,compute_lights).
task(3). name(3,show_controls).
decomp(3,0,{1,2,3},and).
\end{verbatim}
\end{quote}
\item Figure~\ref{fig:examples:clarification:formal}: There are only two accepted arguments. The resulting specification is:
\begin{quote}
\begin{verbatim}
actor(0). name(0, student).
task(1).  name(1,road_pattern).
has(0,1).
\end{verbatim}
\end{quote}
This corresponds to the right-hand GRL model of Figure~\ref{fig:examples:clarification}.
\item Figure~\ref{fig:examples:relevant-actor2:formal}. There are two accepted arguments, but the \emph{generic counterargument} does not contain any formulas. Therefore the resulting specification is:
\begin{quote}
\begin{verbatim}
actor(0).
name(0,dev_team).
\end{verbatim}
\end{quote}
This corresponds to the right-hand GRL model of Figure~\ref{fig:examples:relevant-actor2}.
\end{itemize}