\section{Introduction}
\label{sect:introduction}

Requirements Engineering (RE) is an approach to assess the role of a future information system within a human or automated environment. An important goal in RE is to produce a consistent and comprehensive set of system requirements covering different aspects of the system, such as general functional requirements, operational environment constraints, and so-called non-functional requirements such as security and performance. 

One of the first activities in RE are the ``early-phase'' requirements engineering activities, which include those that consider how the intended system should meet organizational goals, why it is needed, what alternatives may exist, what implications of the alternatives are for different stakeholders, and how the interests and concerns of stakeholders might be addressed~\cite{yu1997towards}. This is generally referred to as goal modeling. Given the number of currently established RE methods using goal models in the early stage of requirements analysis (e.g.,~\cite{liu2004designing,donzelli2004goal,dardenne1993goal,chung2012non,castro2002towards}, overviews can be found in~\cite{van2001goal,kavakliL05}), there is a general consensus that goal models are useful in RE. Several goal modeling languages have been developed in the last two decades. The most popular ones include $i*$~\cite{Yu:1997:TMR:827255.827807}, Keep All Objects Satisfied (KAOS)~\cite{van2008requirements}, the NFR framework~\cite{chung2012non}, \textsc{Tropos}~\cite{giorgini2005goal}, the Business Intelligence Model (BIM)~\cite{horkoff2014strategic}, and the Goal-oriented Requirements Language (GRL)~\cite{Amyot:2010:EGM:1841349.1841356}.

A goal model is often the result of a discussion process between a group of stakeholders. For small-sized systems, goal models are usually constructed in a short amount of time, involving stakeholders with a similar background. Therefore, it is often not necessary to record all of the details of the discussion process that led to the final goal model. 
However, most real-world information systems -- e.g., air-traffic management, industrial production processes, or government and healthcare services -- are complex and are not constructed in a short amount of time, but rather over the course of several workshops. In such situations, failing to record the discussions underlying a goal model in a structured manner may harm the success of the RE phase of system development for several reasons. 

\todo{Marc}{Sepideh}{Please improve 2 and 4 by adding references}

\begin{enumerate}
\item
It is well-known that stakeholders' preferences are rarely absolute, relevant, stable, or consistent~\cite{march1978bounded}. Therefore, it is possible that a stakeholder changes his or her opinion about a modeling decision in between two goal modeling sessions, which may require revisions of the goal model. If previous preferences and opinions are not stored explicitly, it is not possible to remind stakeholders of their previous opinions, thus risking unnecessary discussions and revisions. As the number of participants increases, revising the goal model based on changing preferences can take up a significant amount of time. \todo{Floris}{all}{Depending on whether we use preferences in argumentation models, we might want to rephrase this}
\item
Other stakeholders, who were not the original authors of the goal model, may have to make sense of the goal model, for instance, to use it as an input in a later RE stage. If these user have no knowledge of the underlying rationale of the goal model, it may not only be more difficult to understand the model, but they may also end up having the same discussions as the previous group of stakeholders.
\item
Alternative different ideas and opposing views that could potentially have led to different goal diagrams are lost. For instance, a group of stakeholders specifying a goal model for a user interface may decide to reduce softgoals ``easy to use'' and ``fast'' to one softgoal ``easy to use''. Thus, the resulting goal model will merely contain the softgoal ``easy to use'', but the discussion as well as the decision to reject ``fast'' are lost. This leads to a poor understanding of the problem and solution domain. In fact, empirical data suggest that this is an important reason of RE project failure~\cite{curtis1988field}. 
\item
It is not possible to reason about changing beliefs and opinions, and their effect on the goal model. A stakeholder may change his or her opinion, but it is not always directly clear what its effect is on the goal model. Similarly, a part of the goal model may change, but it is not possible to reason about the consistency of this new goal model with the underlying beliefs and arguments. This becomes more problematic if the participants constructing the goal model change, since modeling decisions made by one group of stakeholders may conflict with the underlying beliefs put forward by another group of stakeholders.
\end{enumerate}
 
The aim of this research is to resolve the above issues by designing a framework, which combines a methodology for the goal modeling process \todo{Floris}{Sepideh}{Do you know of any generic methodologies or heuristics for goal modeling (akin to Juretas GAM)?} with ideas from Artificial Intelligence on practical reasoning and argumentation \cite{atkinson2007} and simple software tool support. We identified several important requirements for our framework: (1) it must be able to formally model parts of the discussion process in the early-requirements phase of an information system, (2) it must be able to generate goal models based on the discussions, (3) it should have formal traceability links between goal elements and arguments in the discussions, (4) it must have tool support, (5) a methodology must be identified that allows the framework to be used by practitioners, and (6) the framework must identify arguments and rationales that might not have been found, or might have been lost, otherwise \todo{Floris}{all}{I think (6) should be replaced with something we actually do in the paper: we don't show that arguments etc. allows us to find rationales that would otherwise not have been found -- for this we would need an experiment. Tailor it towards our empirical work: ``the framework should be close to on the actual discussions stakeholders or designers have in the early requirements engineering phase''}. 

In this context, the main research question is: \emph{What is a suitable framework to formally capture the discussions between stakeholders such that it can generate goal models, and how to implement this framework?} The first five requirements are the success criteria of our approach, that is, the satisfaction of the five requirements will result in a positive answer to the research question. \todo{Sepideh}{all}{The last sentence has to be changed in the end to reflect the result when the paper/evaluation is done. If we write it this way, it means we are unsure of the result.}

\subsection{Contributions} 

The contributions of this paper are aligned with our five success criteria. In order to formalize discussions of the early requirements phase of an information system (requirement 1), we use a technique from argumentation and discourse modelling called \emph{argument schemes} \cite{walton-etal2004}. This allows us to formulate arguments and counter-arguments using so-called \emph{critical questions}. By combining arguments and critical questions, one can create a structured representation of a discussion. We first propose a set of argument schemes and critical questions we expect to find in the discussions that take place during the goal modelling process. We then validate this initial set with transcripts containing discussions about the architecture of an information system. This then gives us the final set of arguments and critical questions.

In order to generate goal models based on formalized discussions (requirement 2), we first formalize the list of arguments from requirement 1 in an argumentation framework. We formalize the critical questions as algorithms modifying the argumentation framework. We use argumentation techniques from AI in order to determine which arguments are accepted and which are rejected. We propose an algorithm to generate a GRL model based on accepted arguments. This creates traceability links from GRL elements to underlying arguments (requirement 3).

We implement our framework into an online tool called RationalGRL (requirement 4). The tool is implemented using Javascript. It contains  two parts, goal modeling and argumentation. The goal modeling part is a simplified version of GRL, leaving out features such as evaluation algorithms and key performance indicators . The argumentation part is new, and we develop a modeling language for the arguments and critical questions. The created GRL models in RationalGRL can be exported to jUCMNav~\cite{}, the Eclipse-based tool for GRL modeling, for further evaluation and  analysis. 

Our final contribution is a methodology on how to develop goal models that are linked to underlying discussions. The methodology consists of two parts, namely argumentation and goal modeling. In the argumentation part, one puts forward arguments and counter-arguments by applying critical questions. When switching to the goal modeling part, the accepted arguments are used to create a goal model. In the goal modeling part, one simply modifies goal models, which may have an effect on the underlying arguments. This might mean that the underlying arguments are no longer consistent with the goal models.

\subsection{Organization}

The first two sections are introductory: Section~\ref{sect:background} contains background and introduces the Goal-oriented Requirements Language (GRL)~\cite{} and argument schemes, while Section~\ref{sect:preview} provides a brief and high-level overview of our framework and methodology using an example. 

The rest of the article is in line with the main contributions. We develop the list of argument schemes in two iterations, corresponding to two sections: In Section~\ref{sect:gmas}, we introduce an initial list of Argument Schemes for Goal Modeling (GMAS) and associated critical questions based on an existing argument scheme. In Section~\ref{sect:gmas:2}, we evaluate the initial list of argument schemes by annotating transcripts from discussions about an information system, which results in our final list of argument schemes and critical questions. In Section~\ref{sect}, we develop a formal model for argument schemes and critical questions and we describe RationalGRL tool in Section~\ref{sect:implementation}. In Section~\ref{sect:methodology}, we propose our methodology while we discuss the related work in Section~\ref{sect:relatedwork}.