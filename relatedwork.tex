\section{Related Work}
\label{sect:relatedwork}

There are several contributions that relate argumentation-based techniques with goal modeling. The contribution most closely related to ours is the work by Jureta \emph{et al.}~\cite{Jureta:RE2008}. This work proposes ``Goal Argumentation Method (GAM)'' to guide argumentation and justification of modeling choices during the construction of goal models. One of the elements of GAM is the translation of formal argument models to goal models (similar to ours). In this sense, our \textsf{RationalGRL} framework can be seen as an instantiation and implementation of  part of the GAM. One of the main contribution of \textsf{RationalGRL} is that it also takes the acceptability of arguments as determined by the argumentation semantics \cite{Dung1995} into account when translating from arguments to goal models.  \textsf{RationalGRL} also provides tool support for argumentation, i.e. Argument Web toolset, to which OVA belongs \cite{bex2013implementing}, and for goal modeling, i.e. jUCMNav~\cite{jUCMNav}. Finally, \textsf{RationalGRL} is based on the practical reasoning approach of \cite{Atkinson2014}, which itself is also a specialization of Dung's~\cite{Dung1995} abstract approach to argumentation. Thus, the specific critical questions and counterarguments based on these critical question proposed by~\cite{Atkinson2014} can easily be incorporated into \textsf{RationalGRL}. 

\textsf{RationalGRL} framework is also closely related to frameworks that aim to provide a design rationale (DR)~\cite{shum2006hypermedia}, an explicit documentation of the reasons behind decisions made when designing a system or artefact. DR looks at issues, options and arguments for and against the various options in the design of, for example, a software system, and provides direct tool support for building and analyzing DR graphs. One of the main improvements of \textsf{RationalGRL} over DR approaches is that \textsf{RationalGRL} incorporates the formal semantics for both argument acceptability and goal satisfiability, which allow for a partly automated evaluation of goals and the rationales for these goals. 

Arguments and requirements engineering approaches have been combined by, among others, Haley \emph{et al.}~\cite{haley2005arguing}, who use structured arguments to capture and validate the rationales for security requirements. However, they do not use goal models, and thus, there is no explicit trace from arguments to goals and tasks. Furthermore, like~\cite{Jureta:RE2008}, the argumentative part of their work does not include formal semantics for determining the acceptability of arguments, and the proposed frameworks are not actually implemented. Murukannaiah \emph{et al.}~\cite{murukannaiah2015resolving} propose Arg-ACH, an approach to capture inconsistencies between stakeholders' beliefs and goals, and resolve goal conflicts using argumentation techniques.