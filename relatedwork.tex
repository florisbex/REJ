\section{Related Work}
\label{sect:relatedwork}

There are several contributions that relate argumentation-based techniques with goal modeling. The contribution most closely related to ours is the work by Jureta \emph{et al.}~\cite{Jureta:RE2008}. This work proposes ``Goal Argumentation Method (GAM)'' to guide argumentation and justification of modeling choices during the construction of goal models. One of the elements of GAM is the translation of formal argument models to goal models (similar to ours). In this sense, our \textsf{RationalGRL} framework can be seen as an instantiation and implementation of  part of the GAM. One of the main contribution of \textsf{RationalGRL} is that it also takes the acceptability of arguments as determined by the argumentation semantics \cite{Dung1995} into account when translating from arguments to goal models.  \textsf{RationalGRL} also provides tool support for argumentation, i.e. Argument Web toolset, to which OVA belongs \cite{bex2013implementing}, and for goal modeling, i.e. jUCMNav~\cite{jUCMNav}. Finally, \textsf{RationalGRL} is based on the practical reasoning approach of \cite{Atkinson2014}, which itself is also a specialization of Dung's~\cite{Dung1995} abstract approach to argumentation. Thus, the specific critical questions and counterarguments based on these critical question proposed by~\cite{Atkinson2014} can easily be incorporated into \textsf{RationalGRL}. 

\textsf{RationalGRL} framework is also closely related to frameworks that aim to provide a design rationale (DR)~\cite{shum2006hypermedia}, an explicit documentation of the reasons behind decisions made when designing a system or artefact. DR looks at issues, options and arguments for and against the various options in the design of, for example, a software system, and provides direct tool support for building and analyzing DR graphs. One of the main improvements of \textsf{RationalGRL} over DR approaches is that \textsf{RationalGRL} incorporates the formal semantics for both argument acceptability and goal satisfiability, which allow for a partly automated evaluation of goals and the rationales for these goals. 

Arguments and requirements engineering approaches have been combined by, among others, Haley \emph{et al.}~\cite{haley2005arguing}, who use structured arguments to capture and validate the rationales for security requirements. However, they do not use goal models, and thus, there is no explicit trace from arguments to goals and tasks. Furthermore, like~\cite{Jureta:RE2008}, the argumentative part of their work does not include formal semantics for determining the acceptability of arguments, and the proposed frameworks are not actually implemented. Murukannaiah \emph{et al.}~\cite{murukannaiah2015resolving} propose Arg-ACH, an approach to capture inconsistencies between stakeholders' beliefs and goals, and resolve goal conflicts using argumentation techniques.

Such arguments, which can be constructed using the online OVA tool\footnote{\url{http://ova.arg-tech.org/}}, can be combined with further arguments providing, for example, expert opinions about the practical reasoning elements (Figure~\ref{fig:pras:example3}, left). Arguments built in OVA can automatically be translated to GRL diagrams using further online tooling\footnote{\url{https://github.com/RationalArchitecture/RationalGRL}} \cite{vanZee-etal:comma2016}. This translation takes the elements of the arguments and directly translates this to GRL elements. Tasks/actions, goals and softgoals/values can be directly translated from the arguments to the GRL diagrams, as can realization statements. Elements of arguments that have nothing to do with goal modelling (e.g. the expert opinion premise in Figure~\ref{fig:pras:example3}) can be included in the GRL diagram as \emph{beliefs}. Finally, attacks between arguments are captured as negative contribution links. Take, for example, \todo{example}. A very similar approach to argumentation and goal models had already been taken by Jureta et al.~\cite{Jureta:RE2008} who proposed a formal argumentation model that can be used to justify goal-modelling decisions. This argumentation model, while not explicitly based on a practical reasoning scheme, is essentially the same as our previous argumentation model based on PRAS: a formal model of structured arguments with goals as premises and tasks as conclusions, and a translation function of these arguments to goal diagrams. 